% License information <<<
%% Copyright 2019 Elsevier Ltd
%% 
%% This file is part of the 'CAS Bundle'.
%% --------------------------------------
%% 
%% It may be distributed under the conditions of the LaTeX Project Public
%% License, either version 1.2 of this license or (at your option) any
%% later version.  The latest version of this license is in
%%    http://www.latex-project.org/lppl.txt
%% and version 1.2 or later is part of all distributions of LaTeX
%% version 1999/12/01 or later.
%% 
%% The list of all files belonging to the 'CAS Bundle' is
%% given in the file `manifest.txt'.
%% 
%% Template article for cas-dc documentclass for 
%% double column output.

%\documentclass[a4paper,fleqn,longmktitle]{cas-dc}
% >>>
\documentclass[a4paper,fleqn]{cas-sc}

%\usepackage[authoryear,longnamesfirst]{natbib}
\usepackage[authoryear]{natbib}
\usepackage{amsmath}
\usepackage{gensymb}
\usepackage{subcaption}
\usepackage{siunitx}
\usepackage{setspace}
%\usepackage{lineno}
%\usepackage[numbers]{natbib}

%%%Author definitions
\def\tsc#1{\csdef{#1}{\textsc{\lowercase{#1}}\xspace}}
\tsc{WGM}
\tsc{QE}
\tsc{EP}
\tsc{PMS}
\tsc{BEC}
\tsc{DE}
%%%

%\linenumbers

\begin{document}
% Author particulars <<<
\let\WriteBookmarks\relax
\def\floatpagepagefraction{1}
\def\textpagefraction{.001}
\shorttitle{Energy Harvesting from an Angled Cruciform}
\shortauthors{A. Adzlan, M.S.M. Ali and S.A. Zaki}

\title [mode = title]{Vortex-induced Vibration of an Angled Cruciform for Energy Harvesting}                      
%\tnotemark[1,2]
%
%\tnotetext[1]{This document is the results of the research
%   project funded by the National Science Foundation.}
%
%\tnotetext[2]{The second title footnote which is a longer text matter
%   to fill through the whole text width and overflow into
%   another line in the footnotes area of the first page.}



\author[1,2]{Ahmad Adzlan}[orcid=0000-0003-0290-3185]
\cormark[1]
%\fnmark[1]
\ead{aafkhairi@graduate.utm.my}
%\ead[url]{www.cvr.cc, cvr@sayahna.org}

\credit{Conceptualisation, Methodology, Software, Validation, Formal analysis, Investigation, Data curation, Writing - Original draft preparation, Visualisation}

\address[1]{Malaysia-Japan International Institute of Technology, Universiti Teknologi Malaysia, 54200 Kuala Lumpur, Malaysia}

\author[1]{Mohamed Sukri Mat Ali}
%\fnmark[1]
\ead{sukri.kl@utm.my}
%\ead[URL]{www.sayahna.org}
\credit{Conceptualisation, Methodology, Resources, Writing - Review \& Editing, Supervision, Project administration, Funding acquisition}

\author[1]{Sheikh Ahmad Zaki}[orcid=0000-0001-6411-9965]
%\fnmark[1]
\ead{sheikh.kl@utm.my}
%\ead[URL]{www.sayahna.org}
\credit{Resources, Writing - Review \& Editing}

\address[2]{Faculty of Engineering, Universiti Malaysia Sarawak, 94300 Kota Samarahan, Sarawak, Malaysia}

\cortext[cor1]{Corresponding author}
%\cortext[cor2]{Principal corresponding author}
%\fntext[fn1]{This is the first author footnote. but is common to third
%  author as well.}
%\fntext[fn2]{Another author footnote, this is a very long footnote and
%  it should be a really long footnote. But this footnote is not yet
%  sufficiently long enough to make two lines of footnote text.}

%\nonumnote{This note has no numbers. In this work we demonstrate $a_b$
%  the formation Y\_1 of a new type of polariton on the interface
%  between a cuprous oxide slab and a polystyrene micro-sphere placed
%  on the slab.
%  }
% >>>
% Macros for commmon symbols <<<
\newcommand{\ypl}{y^{+}} %yPlus
\newcommand{\ured}{U^{*}} %reduced velocity
\newcommand{\yrms}{y^{*}_{\text{RMS}}} %root-mean-square of the normalised cylinder displacement
\newcommand{\ystr}{y^{*}} %the normalised cylinder displacement
\newcommand{\fstr}{f^{*}} %the normalised vibration frequency
\newcommand{\fn}{f_{n}} %system natural frequency
\newcommand{\fk}{f_{k}} %the coarsest grid in a grid independence study
\newcommand{\fcyl}{f_{\text{cyl.}}} %frequency of cylinder vibration
\newcommand{\fosc}{f_{\text{osc.}}} %frequency of cylinder oscillation
\newcommand{\fclstr}{f_{\text{Cl}}^{*}} %normalised frequency of lift coefficient
\newcommand{\flrms}{F_{\text{L,RMS}}} %root-mean-square of the lift force
\newcommand{\fl}{F_{\text{L}}} %the lift force
\newcommand{\clrms}{\text{Cl}_{\text{RMS}}} %root-mean-square of the lift coefficient
\newcommand{\cflyt}{C_{F_{L},y(t)}} %IMF component of lift that is most similar to the displacement signal in terms of temporal evolution of amplitude and frequency, differing only perhaps in phase OR the component of lift with the highest correlation to the displacement signal
\newcommand{\cflkrms}{C_{F_{L},\text{Karman},\text{RMS}}} %the Karman component of lift
\newcommand{\cflsrms}{C_{F_{L},\text{streamwise},\text{RMS}}} %the streamwise component of lift
\newcommand{\ccli}{C_{\text{Cl},i}} %the ith component of lift coefficient
\newcommand{\cclystr}{C_{\text{Cl},\ystr}} %the ith component of lift coefficient
\newcommand{\cflm}{C_{F_{L},\text{max}}} %IMF component of lift that has maximum RMS amplitude in the IMF set
\newcommand{\cyrms}{C_{y,\text{RMS}}} %the RMS of the component of lift that is most correlated with the cylinder displacement signal
\newcommand{\cclrms}{C_{\text{Cl},\text{RMS}}} %the RMS of the component of lift that is most correlated with the cylinder displacement signal (new symbol)
\newcommand{\cysys}{C_{\ystr,\ystr}} %the characteristic IMF representing the normalised cylinder displacement
\newcommand{\cclys}{C_{\text{Cl},\ystr}} %the characteristic IMF representing the lift coefficient
\newcommand{\afl}{\alpha_{F_{L}}} %ratio between two dominant IMF components of the lift
\newcommand{\pfrms}{P_{\text{Fluid,RMS}}} %estimated root-mean-square of fluid power
\newcommand{\pmrms}{P_{\text{Mech.,RMS}}} %estimated root-mean-square of mechanical power
\newcommand{\re}{\text{Re}} %Reynolds number
\newcommand{\st}{\text{St}} %Strouhal number
\newcommand{\phim}{\phi_{\text{mean}}} %mean phase lag
\newcommand{\wcl}{W_{\text{cyl.}}} %mean work done by cylinder over one cycle of vibration
\newcommand{\tosc}{T_{\text{osc.}}} %mean period of cylinder oscillation
\newcommand{\meff}{m_{\text{eff.}}} %effective mass
\newcommand{\zetatot}{\zeta_{tot.}} %total damping of the system

%Macros that are shorthands in writing
\newcommand{\rms}{root-mean-square} %shorthand for root-mean-square

%Macros used in writing section on GCI study
\newcommand{\rp}{r^{p}} %refinement ratio, used in GCI study
\newcommand{\fre}{f_{\text{RE}}} %Richardson extrapolation of quantity of interest, used in GCI study

%The macros for freestream velocities
\newcommand{\uon}{\SI[per-mode=symbol]{0.1}{\metre\per\second}}
\newcommand{\utw}{\SI[per-mode=symbol]{0.2}{\metre\per\second}}
\newcommand{\uth}{\SI[per-mode=symbol]{0.3}{\metre\per\second}}
\newcommand{\ufo}{\SI[per-mode=symbol]{0.4}{\metre\per\second}}
\newcommand{\ufi}{\SI[per-mode=symbol]{0.5}{\metre\per\second}}
\newcommand{\usi}{\SI[per-mode=symbol]{0.6}{\metre\per\second}}
\newcommand{\use}{\SI[per-mode=symbol]{0.7}{\metre\per\second}}
\newcommand{\uei}{\SI[per-mode=symbol]{0.8}{\metre\per\second}}
\newcommand{\uni}{\SI[per-mode=symbol]{0.9}{\metre\per\second}}
\newcommand{\ute}{\SI[per-mode=symbol]{1.0}{\metre\per\second}}
\newcommand{\uel}{\SI[per-mode=symbol]{1.1}{\metre\per\second}}
\newcommand{\utv}{\SI[per-mode=symbol]{1.2}{\metre\per\second}}
\newcommand{\utt}{\SI[per-mode=symbol]{1.3}{\metre\per\second}}

\newcommand{\uron}{2.3}
\newcommand{\urtw}{4.5}
\newcommand{\urth}{6.8}
\newcommand{\urfo}{9.1}
\newcommand{\urfi}{11.4}
\newcommand{\ursi}{13.6}
\newcommand{\urse}{15.9}
\newcommand{\urei}{18.2}
\newcommand{\urni}{20.5}
\newcommand{\urte}{22.7}
\newcommand{\urel}{25.0}
\newcommand{\urtv}{27.3}
\newcommand{\urtt}{29.5}
% >>>
\begin{abstract}
  We investigated the displacement and lift time series of a circular cylinder - strip plate cruciform system for energy harvesting in the Reynolds number range $1.1 \times 10^{3} \leq \text{Re} \leq 14.6 \times 10^{3}$, numerically using the open source C++ library: OpenFOAM. The Karman vortex-induced vibration (KVIV) regime was identified between reduced velocity, $U^{*}$, $2.3$ and $13.6$, while the streamwise vortex-induced vibration (SVIV) regime was identified between $18.2 \leq U^{*} \leq 29.5$. We analysed the cylinder displacement and lift time series using the Hilbert-Huang transform (HHT). Within this range of $U^{*}$, Karman vortex shedding contributes nearly as much as streamwise vortex shedding to the root-mean-square amplitude of total lift, while between $25.0 \leq U^{*} \leq 29.5$, the Karman component contribution is on average twice that of the streamwise component. These findings hint at the possibility to improve the power output of the harvester by a factor of two between $18.2 \leq U^{*} \leq 22.7$ and by a factor of three between $25.0 \leq U^{*} \leq 29.5$, if we can unite the contribution to the root-mean-square amplitude of the total lift under a single vibration-driving mechanism: the shedding of streamwise vortex.
\end{abstract}
%%fakesection GraphicalAbstract
\begin{graphicalabstract}
  \includegraphics[width=0.75\textwidth]{figs/graphicalAbstract}
\end{graphicalabstract}

%%fakesection Highlights
\begin{highlights}
\item Three main energy harvesting regimes were identified, based on the cruciform angle
\item The streamwise-vortex induced vibration regime produces power in the order of \SI{1}{\milli\watt}
\item Power output is below \SI{1}{\milli\watt} between angles \SI{67.5}{\degree} and \SI{22.5}{\degree}
\item Power output can achieve up to \SI{10}{\milli\watt}, as the angle is brought closer to \SI{0}{\degree}.
\end{highlights}
\begin{keywords}
  Vortex-induced vibration \sep Vibration energy harvester \sep CFD simulation \sep Streamwise vorticity \sep Ensemble empirical mode decomposition (EEMD) \sep Hilbert transform
\end{keywords}


\maketitle

\doublespacing

\section{Introduction} \label{sec:intro}
Streamwise vortex-induced vibration (SVIV) is a type of vortex-induced vibration (VIV) driven by vortical structures whose vorticity vector points in the direction of the free stream. In recent decades, there have been efforts to exploit the SVIV phenomenon from cruciform structures for energy harvesting, an example of which is given in Fig. The literature on this subject can be broadly categorised into two groups: how the mechanical properties of the oscillator (e.g., mass ratio, damping, etc.) affects the amplitude/frequency response of SVIV \citep{Koide2009,Koide2013,Nguyen2012} and how the minutiae of the flow field affect the force driving the vibration of the cylinder, i.e. the fluid mechanical aspect of the system \citep{Deng2007,Koide2017,Zhao2018a}.

In the first focus area, researchers studied some permutation of the following method to convert the vibration into electrical power. The method consists of a coil and magnet. The coil, which moves with the vibrating cylinder, creates relative motion against the magnet, which is placed in the hollow of the coil \citep{Koide2009}. While investigating the system at a Reynolds number in the order of $\re \sim O \left( 10^{4} \right)$, \citet{Koide2009} showed that increased damping due to energy harvesting reduces the maximum vibration amplitude close to a factor of 4. Amplitude reduction due to increased total damping was also mentioned in \citet{Bernitsas2008a,Bernitsas2008b,Bernitsas2009}. Further investigation in \citet{Nguyen2012} revealed that damping not only affects the amplitude response of the cylinder but also narrows the synchronisation region between vortex shedding and cylinder vibration. Moreover, \citet{Nguyen2012} demonstrated a strong coupling between mass ratio and damping in determining both the width of the synchronisation region and the maximum amplitude response of the cylinder.

In the second focus area, investigators turned their attention to the details of the flow where streamwise vortex shedding occurs. One such study carefully shot motion pictures of the dye-injected flow \citep{Koide2017} at Reynolds number in the order of $\re \sim O\left( 10^{3} \right)$. A lower Reynolds number (Re) reduces the amount of turbulence in the flow, allowing a clearer shot of the vortex structures. Their study also highlights the higher level of turbulence produced by the circular cylinder-strip plate cruciform in contrast to the twin circular cylinder cruciform, which diminishes the periodicity of vortex shedding. Although visually enlightening, this and other more qualitative studies contribute little towards improving our understanding of the relationship between vortex shedding and the resulting lift. \citet{Deng2007} demonstrated a way to overcome such a shortcoming.

In their study, \citet{Deng2007} examined the flow field of a twin circular cylinder cruciform using computational fluid dynamics (CFD). Their domain stretches  $28D$  in the streamwise direction,  $16D$  in the transverse direction and  $12D$  in the spanwise direction. They studied an Re range yet another order of magnitude smaller than that studied by \citet{Koide2017}, possibly to get an even clearer visualisation of the vortical structures with less turbulence, and to ease computational requisites. At a fixed  $\re = 150$ , streamwise vortices form even at a gap ratio of $2$. This result differs quite strikingly from \citet{Koide2006,Koide2007}, conducted at an Re twice the order of magnitude of \citet{Deng2007}, an indication that the minimum gap ratio needed for the onset of streamwise varies with respect to Re.

They also observed that when the gap ratio $G$, which they denote as  $L/D$  in their paper, increases from 3 to 4, the maximum amplitude of the lift coefficient increases by almost threefold. This can be attributed quite easily to the current vortex pair shed by the upstream cylinder. The downstream cylinder immediately disturbs the pair shed from the upstream cylinder when  $G=3$. The lift coefficient increases by about a factor of 3 when this immediate disturbance diminishes at  $G=4$. The visualisation of three-dimensional (3D) vorticity isocontours enables us to quickly establish this link vis-\`{a}-vis the lift coefficient signal. The authors use of CFD made this possible.

A similar study in the order of magnitude $\re \sim O \left( 10^{2} \right)$ by \citet{Zhao2018a} particularly highlighted the immense utility of CFD as a tool to research SVIV or flow around a cruciform in general. They computed the sectional lift coefficient along the upstream cylinder, and the time history of this sectional lift coefficient revealed two different modes of vortex shedding, namely, parallel and K-shaped. They also paid attention to the local flow patterns that vary along the length of the upstream cylinder such as the trailing vortex flow, necklace vortex flow and flow in the small gap (denoted as SG flow). The discontinuities in the phase angle of the sectional lift coefficient along the upstream cylinder seems to suggest the inadequateness of attributing the lift coefficient to streamwise vortex shedding alone, particularly when Karman vortex streamlines were also observed some distance away from the junction of the cruciform. \citet{Shirakashi1989} also made a similar observation in their experimental work. This leads us to hypothesise that the lift signal is more appropriately viewed as the streamwise-Karman vortex-induced composite lift signal. However, we could not find studies that took this viewpoint and worked out its implication on power generation in their investigation of SVIV.

The objectives of this study are thus threefold: (1) to take a closer look at the amplitude and frequency response of a circular cylinder-strip plate cruciform, especially in reduced velocity ($\ured$) ranges where the transition from KVIV to SVIV occurs, (2) to demonstrate the compositeness of the lift signal of an SVIV system and establish the difference between the lift signal characteristics in the KVIV and SVIV regime and (3) to shed light on how the contribution from the Karman and streamwise components of lift changes as we increase $\ured$ after the onset of SVIV and predict how much improvement in the power generation can be anticipated if we are able to unify the lift amplitude contributions due to Karman and streamwise vortex shedding. Here, $\ured = U/\fn D$, with $U$, $\fn$ and $D$ being the freestream velocity, natural frequency of the system and the diameter of the circular cylinder respectively. The following \S\ref{sec:method} details the methodology we employ to conduct this study. We present and discuss our results in. We describe our conclusions in \S\ref{sec:conclusions}.

\begin{figure}
  \centering
  \includegraphics[width=0.5\textwidth]{figs/oscillatorSchematic}
  \caption{Schematic of the base configuration of the oscillator system used in this study, i.e. the pure cruciform. In this configuration, the axis of the cylinder and the strip plate are perpendicular to each other.}
  \label{fig:oscillatorSchematic}
\end{figure}

\section{Methodology} \label{sec:method}
\subsection{Problem geometry} \label{ssec:probGeo}
This study bases itself on the work done by \citet{Maruai2017}, \citet{Maruai2018}, and \citet{Koide2013}. In these works, the investigators conducted both experimental and computational investigations of passive control of FSI of cylinders using a strip plate located at the cylinder downstream. Here, the term ``strip plate'' is used as a shorthand for the long, rectangular plate used to control the vibration of the cylinder -- since the plate resembles a strip due to its large aspect ratio. These studies demonstrated the feasibility of energy harvesting using the oscillator system described, in the Reynolds number range $3.6\times10^{3}<\text{Re}<12.5\times10^{3}$. Following this observation, we performed our numerical investigations within a similar \re{} range, albeit slightly widened, to check for variations in the cylinder response in as wide an \re{} range as possible, computational resources permitting.

Our oscillator system derives its geometry from the works of \citet{Nguyen2012}, \citet{Koide2013}, and \citet{Koide2017}. The basic layout of our oscillator system is the pure cruciform: an arrangement where the circular cylinder and the strip plate located downstream have their axes perpendicular to each other. We fixed the gap between the cylinder and the strip plate, $G$, to $0.16$. This value of $G$ was chosen because the cylinder response most suitable for energy harvesting is sustained over the largest range of reduced velocity $\ured$ when $G = 0.16$ \citep{Koide2013}.

\begin{figure}
  \centering
  \begin{subfigure}[h]{0.5\textwidth}
    \includegraphics[width=\textwidth]{figs/problemGeometrySide}
    \caption{}
    \label{fig:probGeoSide}
  \end{subfigure}

  \begin{subfigure}[h]{0.5\textwidth}
    \includegraphics[width=\textwidth]{figs/problemGeometryTop}
    \caption{}
    \label{fig:probGeoTop}
  \end{subfigure}

  \caption{Figure \ref{fig:probGeoSide} shows the cross-sectional layout of the computational domain, along with key dimensions, when viewed from the side. Figure \ref{fig:probGeoTop} visualises the cross-section of the computational domain as viewed from the top. The arbitrarily coupled mesh interface (ACMI) used to connect the domain containing the cylinder with the domain containing the strip plate is placed halfway through the gap, i.e. $0.08D$ downstream the cylinder.} \label{fig:problemGeometry}
\end{figure}

Figure \ref{fig:probGeoSide} visualises our computational domain from its side. We chose these dimensions based on analogous works such as \cite{Maruai2017} and \cite{Maruai2018} which had produced results that agree well with experiments of their own and with \citet{Kawabata2013}. The streamwise coordinates of this domain extends from $-10.5D$ to $10.5D$, and the lateral coordinates from $-10.5D$ to $10.5D$. The coordinate origin $(0,0,0)$, is at the centre of the cylinder and the strip plate is $D/3$ thick.

In Fig. \ref{fig:probGeoTop}, the circular cylinder extends from $z/D=7.5$ to $z/D=-7.5$, giving the computational domain an overall spanwise length of $15D$. The computational domain of a similar study by \citet{Deng2007} has a length of $12D$, upon which the dimension of our domain is based upon. The extra $1.5D$ of spanwise length on either side of our domain is allocated to ensure the full expression of the three-dimensionality of flow structures that appear during the course of our numerical study.

\begin{figure}
  \centering
  \begin{subfigure}[h]{0.3\textwidth}
    \includegraphics[width=\textwidth]{figs/cruciform90}
    \caption{Cruciform layout for \SI{90}{\degree}}
    \label{fig:cruciform90}
  \end{subfigure}
  \hfill
  \begin{subfigure}[h]{0.3\textwidth}
    \includegraphics[width=\textwidth]{figs/cruciform675}
    \caption{Cruciform layout for \SI{67.5}{\degree}}
    \label{fig:cruciform675}
  \end{subfigure}
  \hfill
  \begin{subfigure}[h]{0.3\textwidth}
    \includegraphics[width=\textwidth]{figs/cruciform45}
    \caption{Cruciform layout for \SI{45}{\degree}}
    \label{fig:cruciform45}
  \end{subfigure}

  \raggedright
  \begin{subfigure}[h]{0.3\textwidth}
    \includegraphics[width=\textwidth]{figs/cruciform225}
    \caption{Cruciform layout for \SI{22.5}{\degree}}
    \label{fig:cruciform22.5}
  \end{subfigure}
  \hspace{6mm}
  \begin{subfigure}[h]{0.3\textwidth}
    \includegraphics[width=\textwidth]{figs/cruciform00}
    \caption{Cruciform layout for \SI{0}{\degree}}
    \label{fig:cruciform00}
  \end{subfigure}

  \caption{Variation of cruciforms studied in this work. We vary the cruciform angle from the case of a pure cruciform (\SI{90}{\degree}) to the case of cylinder - plate in tandem (\SI{0}{\degree}), in increments of \SI{22.5}{\degree}.}\label{fig:cruciformLayouts}
\end{figure}

We then produce variants of the pure cruciform configuration by rotating the strip plate from \SI{90}{\degree} to \SI{0}{\degree} in \SI{22.5}{\degree} increments. In total, we constructed five different cruciforms shown in Fig. \ref{fig:cruciformLayouts}. The dimensions of the computational domain remain fixed for all cruciforms, including the gap between the cylinder and the strip plate.

\subsection{Numerical method} \label{ssec:numMeth}
Our numerical study utilises OpenFOAM, an open-source computational fluid dynamics (CFD) platform written in C++. With OpenFOAM, we solved the 3D unsteady Reynolds averaged Navier-Stokes (3D URANS) equations that are the following.

\begin{equation}
  \frac{\partial U_{i}}{\partial x_{i}}=0,
  \label{eq:continuity}
\end{equation}

\begin{equation}
  \frac{\partial U_{i}}{\partial t}+U_{j}\frac{\partial U_{i}}{\partial x_{j}} = -\frac{1}{p}\frac{P}{x_{i}}+\frac{\partial}{\partial x_{j}} \left( 2\nu S_{ij}-\overline{u'_{j}u'_{i}} \right).
  \label{eq:navier-stokes}
\end{equation}

The symbols $U$, $x$, $t$, $\rho$, $P$, $\nu$, $S$, and $u'$ denote the mean component of velocity, spatial component, time, density, pressure, kinematic viscosity, mean strain rate and the fluctuating component of velocity, respectively. Equation \ref{eq:sij} gives the mean strain rate $S_{ij}$.

\begin{equation}
  S_{ij} = \frac{1}{2} \left( \frac{\partial U_{i}}{\partial x_{j}} + \frac{\partial U_{j}}{\partial x_{i}} \right).
  \label{eq:sij}
\end{equation}

The turbulence model employed to approximate the Reynolds stress tensor is the Spalart-Allmaras turbulence model. Previous numerical studies on energy harvesting from FIM of circular cylinders have shown reasonable agreement with experiments in the literature through the use of this turbulence model, and thus becomes the basis for the implementation of the same turbulence model in our study \citep{Ding2015a,Ding2015b}. The Boussinesq approximation relates the Reynolds stress tensor $\tau_{ij} = \overline{u'_{j}u'{i}}$ to the mean velocity gradient, exemplified by Eq. \ref{eq:tauij}.

\begin{equation}
  \tau_{ij} = 2 \nu_{T}S_{ij},
  \label{eq:tauij}
\end{equation}

\noindent where $\nu_{T}$ represents the kinetic eddy viscosity. This kinetic eddy viscosity is ultimately expressed as a function whose arguments consist of the molecular viscosity $\nu$, and an intermediate variable $\tilde{\nu}$ that is the solution of Eq. \ref{eq:kineticEddyTransport}. Equation \ref{eq:kineticEddyTransport} incorporates empirically obtained constants to provide closure to the equations governing our numerical investigation. We list the empirical constants that make up Eq. \ref{eq:kineticEddyTransport} in Table \ref{tab:spalart-Allmaras}.

\begin{equation}
  \label{eq:kineticEddyTransport}
  \frac{\partial \tilde{\nu}}{\partial t} + U_{j} \frac{\partial \tilde{\nu}}{\partial x_{j}} = c_{b1}\tilde{S}\tilde{\nu} - c_{w1} f_{w} \left( \frac{\tilde{\nu}}{D} \right)^{2} + \frac{1}{\sigma} \left\{ \frac{\partial}{\partial x_{j}} \left[ \left( \nu + \tilde{\nu} \right) \frac{\partial \tilde{\nu}}{\partial x_{j}} \right] c_{b2} \frac{\partial \tilde{\nu}}{\partial x_{i}} \frac{\partial \tilde{\nu}}{\partial x_{i}} \right\}
\end{equation}

\begin{table}[width=0.6\textwidth,cols=2,pos=h]
  \caption{Empirical constants used in the Spalart-Allmaras turbulence model.} \label{tab:spalart-Allmaras}
  \begin{tabular*}{\tblwidth}{@{} LL@{} }
    \toprule
      Empirical constants & Value    \\
    \midrule
      $c_{b1}$            & $0.01$   \\
      $c_{b2}$            & $0.09$   \\
      $c_{\nu1}$          & $0.01$   \\
      $\kappa$            & $0.1$    \\
      $\sigma$            & $0.162$  \\
      $c_{\omega3}$       & $0.178$  \\
    \bottomrule
  \end{tabular*}
\end{table}

\noindent We refer the interested reader to the original paper by \citet{Spalart1992} and more recent applications of the turbulence model in \citet{Ding2019} and \citet{Sun2019b}. With the turbulence model properly defined, we are finally able to solve Eqs. \ref{eq:continuity} and \ref{eq:navier-stokes} using the SIMPLE-stabilised PISO algorithm native to OpenFOAM, known as the PIMPLE algorithm. 

\subsection{Dynamic mesh motion} \label{ssec:dynMesh}

Cylinder motion in the computational domain due to FIV introduces distortion to the mesh immediately surrounding the cylinder. The simplest way to keep the mesh distortion in check, thus keeping mesh quality within an acceptable level, is by diffusing the amount of warping to the surrounding space. In practice, the surrounding space is the rest of the mesh nodes, and Eq. \ref{eq:laplace} governs the diffusion.

\begin{equation}
  \nabla \cdot \left( \gamma \nabla u \right) = 0.
  \label{eq:laplace}
\end{equation}

In Eq. \ref{eq:laplace}, $u$ and $\gamma$ represents the mesh deformation velocity and displacement diffusion, respectively. In this work, we set the displacement to be diffused according to the inverse quadratic rule $\gamma = 1/l^{2}$. Here, $l$ denotes the distance from the cell centre to the nearest cylinder edge. Then, we solve Eq. \ref{eq:laplace} using the GAMG algorithm and the Gauss-Seidel smoother. Solution of Eq. \ref{eq:laplace} returns an updated value of $u$, and this updated value of $u$ is used to update the position of the mesh nodes according to Eq. \ref{eq:meshNodeUpdate}. The PIMPLE solver resumes the solution of the 3D URANS equations after we update the mesh node positions.

\begin{equation}
  x_{\text{new}} = x_{\text{old}} + u \Delta t
  \label{eq:meshNodeUpdate}
\end{equation}

For most numerical studies of FSI, the mesh warp diffusion method governed by Eq. \ref{eq:laplace} serves as an adequate workaround to conserve mesh quality. However, this requires ample number of ambient mesh nodes acting as the receiving end of the diffusion algorithm. In our case, the small gap between the cylinder and strip plate ($G = 0.16$) pose a serious limitation to our ability to diffuse the amount of warp introduces by the displacement of the cylinder, since a small space means that we can only allocate a proportionate number of mesh nodes in said gap. Sole reliance on the warp diffusion algorithm will hamper our effort to preserve mesh quality as a high concentration of warp remains within the gap. To overcome this problem, we implement the arbitrarily coupled mesh interface (ACMI) halfway through the gap (see Fig. \ref{fig:problemGeometry}). This technique allows adjacent cells to slide over each other precisely at the $x = 0.13$ plane, ridding us of the requirement for mesh warp diffusion. In the literature, ACMI is also known as the generalised grid interface, or GGI \citep{Zhang2018,Sun2019b}.

\subsection{Open flow channel experiment} \label{ssec:openFlowExp}

As part of the validation process for our numerical setup, we constructed a closed loop open flow channel, with a test section \SI{100}{\milli\metre} wide, \SI{200}{\milli\metre} high and \SI{1500}{\milli\metre} long. The design of this open flow channel is heavily inspired by the water tunnel of \citet{Nguyen2012} and \citet{Koide2013}. Considering the application of this research in the far future is in open flows such as natural drainage systems or the ocean - and not within pipes - prompts us to make this distinction.

We benchmark the open flow channel by setting up a pure cruciform oscillator (\SI{90}{\degree}) experiment, whose data from similar studies are readily available in published works. Following this, we dimensioned the rig to follow the parameters used in \citet{Koide2013}. A summary of our parameters and those used in \citet{Koide2013} are provided in Table \ref{tab:expParameter}. We tune the parameters governing the amplitude/frequency response of the oscillator using simple length-based mechanism as follows (see Fig. \ref{fig:rigSketch}). To tune the spring coefficient $k$, we simply adjust the active length of the twin spring plate. In practice, we obtained the calibration curve of the twin spring plate by performing a weight - displacement measurement \citep{Sun2016} at several active lengths of the plate. Once the spring coefficient versus spring plate active length calibration curve is obtained, we can just adjust the length of the spring plate to achieve the desired value of $k$.

\begin{figure}
  \centering
  \begin{subfigure}[h]{0.5\textwidth}
    \includegraphics[width=\textwidth]{figs/rigSketch}
    \caption{}
    \label{fig:rigSketch}
  \end{subfigure}

  \begin{subfigure}[h]{0.35\textwidth}
    \includegraphics[width=\textwidth]{figs/damperSketch}
    \caption{}
    \label{fig:damperSketch}
  \end{subfigure}

  \caption{Our experimental system used to validate our numerical study. Figure \ref{fig:rigSketch} presents a 3D schematic of the open channel test section with a pure cruciform oscillator setup, while Fig. \ref{fig:damperSketch} shows a magnified schematic of the damping system.} \label{fig:experimentalSetup}
\end{figure}

Tuning the total damping of the system and consequently the multiple expressions of damping such as the logarithmic damping $\delta$, Scruton number Sc, or the damping coefficient is done by attaching, as shown in Fig. \ref{fig:rigSketch}, a T-shaped plate made from aluminium into a claw-shaped casing that houses neodymium magnets at its ends. As presented in Fig. \ref{fig:damperSketch}, the method we use to control the strength of the magnetic field exposed to the T-shaped plate is by fixing the insertion depth of the T-shaped plate into the casing. The magnetic field serves to dissipate the kinetic energy of the T-shaped plate that moves with the cylinder during FIM, providing system damping.

\begin{table}[width=0.9\linewidth,cols=3,pos=h]
  \caption{Summary of experimental parameters in contrast to those used in the experimental work of \citet{Koide2013}.} \label{tab:expParameter}
\begin{tabular*}{\tblwidth}{@{} LLL@{} }
\toprule
                                           & Current study & \citet{Koide2013}\\
\midrule
Cylinder diameter, $D$ (m)                 & $0.01$        & $0.01$           \\
Cylinder length, $l_{\text{cylinder}}$ (m) & $0.09$        & $0.098$          \\
Strip-plate width (m)                      & $0.01$        & $0.01$           \\
Strip-plate length (m)                     & $0.1$         & $0.1$            \\
Effective mass, $m_{\text{eff.}}$ (kg)     & $0.162$       & $0.174$          \\
Logarithmic damping, $\delta$              & $0.178$       & $0.24$           \\
Scruton number, Sc                         & $9.94$        & $7.74$           \\
System natural frequency, $f_{n}$ (Hz)     & $4.42$        & $4.4$ to $4.79$  \\
\bottomrule
\end{tabular*}
\end{table}

A voltage controller regulates the power driving the $3.728$ kW (5 hp) centrifugal pump. To set the freestream velocity in the open flow channel, we placed an acoustic Doppler velocimeter (ADV) sampling at in an empty test section, filled with plain tap water to a height of \SI{100}{\milli\metre}, on the centreline of the channel, as pictured in Fig. \ref{fig:keyDimensions}. The height of \SI{100}{\milli\metre} is also the water level we conduct our experiments in. We keep the water level at this height of \SI{100}{\milli\metre} during all data collections to achieve a flow ambience analogous to our benchmark study of \citet{Koide2013}, facilitating comparison between the two. Then, we sampled the velocity of the flow at different input voltages by the voltage controller, the final product being an input voltage $V_{\text{in}}$ (V) versus centreline velocity $U_{\text{cent.}}$ calibration curve. This calibration curve allows us to set the freestream velocity of the open flow channel by specifying the input voltage to the pump. The finished product gave an operability range between \uth{} and \uel{}, which translates to a $\urth \leq \ured \leq \urel$ for an elastically supported circular cylinder system of diameter \SI{10}{\milli\metre}. The turbulence level ranges between $5\%$ to $8\%$ when the freestream velocity $U_{\infty} \geq \uei$.

\begin{figure}
  \centering
  \includegraphics[width=0.5\textwidth]{figs/keyDimensions}
  \caption{The side view of our test section. For a more valid benchmarking of our open channel flow with a similar system in \citet{Koide2013}, we keep the water level to \SI{100}{\milli\metre}.}
  \label{fig:keyDimensions}
\end{figure}

We measured the cylinder displacement $y$ as a function of time by placing a visual marker on the support plate of the cylinder (see Fig. \ref{fig:rigSketch}) and capturing the motion of the marker using a video camera positioned perpendicular to the support plate. The motion of the marker is then analysed using \textit{Tracker}: a motion analysis tool built on the Open Source Physics Java framework (for recent implementation examples, see \citet{Wen2020}  or \citet{Krishnendu2020}).

For the benchmarking, we chose the reduced velocity $\ured = \urte$, as the cylinder at that $\ured$ produces a large and stable displacement that simplifies on our part, the measurement and comparison process between our experimental system and \citet{Koide2013}. A sample of the normalised displacement -- $\ystr = y/D$ -- measured as a function of time is illustrated in Fig. \ref{fig:sampTimeHist}. This time series allows us to also compute the normalised cylinder vibration frequency, $\fstr = \fcyl/\fn$ ($\fcyl$ being the vibration frequency of the cylinder). The $\ystr$ data presented in Fig. \ref{fig:sampTimeHist} returns $\ystr = 0.33 \pm 0.03$ and $\fstr 1.03 \pm 0.04$, after computing the uncertainty from multiple experimental runs. In their work, \citet{Koide2013} obtained $\ystr = 0.32$ and $\fstr = 1.09$ at a similar $\ured$ - values that are well within the measurement uncertainty of our experiment. This provides a basis for our reliance on results obtained from the experimental system later in the study.

\begin{figure}
  \centering
  \includegraphics[width=0.41\textwidth]{figs/figure5}
  \caption{The normalised cylinder displacement measured as a function of time at $\ured = \urte$. The experiment was repeated several times to estimate the uncertainty of the measured quantities $\ystr$ and $\fstr$.}
  \label{fig:sampTimeHist}
\end{figure}

\section{Numerical setup validation} \label{sec:numSetup}
\subsection{Grid independency study via Richardson extrapolation and grid convergence index} \label{ssec:richExtrap}
Like the first, the second method solves the governing equations on successively finer grids. However, instead of arguing that one obtains similar results on all the grids, the investigator checks whether the quantities of interest tend towards value, as one solves the governing equation on successively finer grid resolutions \citep{Richardson1927,Stern2001}. This method, of checking for convergence pays attention not only on the presumed converged value but also on the trend of convergence. Literature that employ this method impose a monotonic convergence condition \citep{Stern2001,MatAli2011,Ali2012,Maruai2018} on their quantities of interest, adding an extra layer of confidence in the final form of their spatial discretisation.

Additionally, this method allows for a quantitative description of the degree of convergence through the grid convergence index (GCI). Let $f_{1},f_{2},f_{3},\dots,\fk$ denote the quantity of interest obtained from several grids. A larger subscript indicates a coarser grid, thus, $f_{1}$ denotes the finest while $\fk$ denotes the coarsest grid. Let the difference between successive solutions be $\epsilon_{2,1},\epsilon_{3,2},\epsilon_{4,3},\dots,\epsilon_{n,n-1}$, where $\epsilon_{2,1} = f_{2} - f_{1}$, $\epsilon_{3,2} = f_{3} - f_{2}$ and so on. Then, the GCI is defined as

\begin{equation}
  \text{GCI}_{i+1,i} = F_{s} \frac{\left |\epsilon_{i+1,i} \right |}{f_{i} \left ( r^{p} - 1 \right )} \times 100\%,
  \label{eq:gci}
\end{equation}

\noindent where $F_{s}$, $f_{i}$ and $r^{p}$ denotes the safety factor $\left ( = 1.25 \right )$, quantity of interest and the refinement ratio, $r$, between successive grids raised to the order of accuracy of the series of solution, $p$. We refer the reader to \citet{Stern2001,Langley2018} for a more detailed discussion on $r^{p}$.

We can estimate what the solution approaches as the grid size approaches zero by using the $\text{p}^{\text{th}}$ method. Briefly, we compute the generalised Richardson extrapolation of the quantity of interest as follows.

\begin{equation}
  \fre = f_{1} + \frac{f_{1} - f_{2}}{\rp - 1},
  \label{eq:richardsonExtrapolation}
\end{equation}

\noindent where $\fre$ is the Richardson extrapolation of the quantity of interest. Using $\fre$ to estimate the limit of the monotonically convergent series of $f_{i}$, we can determine the percentage difference of our solution on our finest grid from this limit as

\begin{equation}
  E_{i} = \frac{f_{i} - \fre}{\fre} \times 100\%.
  \label{eq:percentageDifference}
\end{equation}

Table \ref{tab:gridIndependency} summarises the result of our grid independency study for the SVIV reduced velocity of $\ured = 22.7$. We identified three quantities central to the investigation of fluid-structure phenomena, especially the flow-induced vibration of a circular cylinder. They are the vibration amplitude, vibration frequency and lift coefficient of the cylinder. We solve the governing equations on three grids which are numbered $1$ for the finest, $2$ for the medium and $3$ for the coarsest, shown in Fig. \ref{fig:convergenceStudy}. If we let $v_{i}$ be the volume of the $i^{\text{th}}$ cell in the grid and $N$ be the total number of cells in the domain, then, the average cell size is


\begin{figure}
  \centering
  \begin{subfigure}[h]{0.3\textwidth}
    \includegraphics[width=\textwidth]{figs/figure6a}
    \caption{Coarse}
    \label{fig:coarseMesh}
  \end{subfigure}

  \begin{subfigure}[h]{0.3\textwidth}
    \includegraphics[width=\textwidth]{figs/figure6b}
    \caption{Medium}
    \label{fig:mediumMesh}
  \end{subfigure}

  \begin{subfigure}[h]{0.3\textwidth}
    \includegraphics[width=\textwidth]{figs/figure6c}
    \caption{Fine}
    \label{fig:fineMesh}
  \end{subfigure}

  \caption{Three meshes used in the grid convergence study. Figures \ref{fig:coarseMesh}, \ref{fig:mediumMesh} and \ref{fig:fineMesh} show the coarse, medium and fine meshes viewed perpendicular to three main viewing positions: from the inlet, the top and the front, which is looking directly at the cylinder end.} \label{fig:convergenceStudy}
\end{figure}

\begin{equation}
  h = \frac{1}{N} \left [ \sum_{i=1}^{N} v_{i} \right ]^{1/3},
  \label{eq:averageCellSize}
\end{equation}

\noindent and the normalised average cell size is hence 


\begin{equation}
  h/D = \frac{1}{ND} \left [ \sum_{i=1}^{N} v_{i} \right ]^{1/3}.
  \label{eq:normAveCellSize}
\end{equation}

Both $\yrms$ and $\clrms$ starts at an initial value smaller than their Richardson extrapolations, $\fre$, before approaching it as we decrease the average cell size, $h$. This similar trend can perhaps be attributed to the causal relationship between the lift coefficient and vibration amplitude. The lift drives and sustains the vibration, hence a small lift produces a small vibration, and when the lift amplitude becomes higher, so too does the vibration amplitude. The vibration frequency, on the other hand, starts at a value larger than its $\fre$ before approaching $\fre$.

The quantity $\clrms$ experiences the most significant drop in GCI as we refine the grid. The GCI is close to one-third $\left ( 30.92\% \right )$ as we refine the grid from coarse to medium with a refinement ratio of $1.376$. The refinement ratio is calculated by dividing the number of cells in one grid with the next one down the refinement line. Following the grid numbering convention explained previously, dividing the number of cells in the fine grid (grid 1) with the number of cells in the medium grid (grid 2) gives us the refinement ratio from medium to fine, or $r_{2,1}$. Similarly, dividing the number of cells in the medium grid (grid 2) with the number of cells in the coarse grid (grid 3) gives us the refinement ratio from coarse to medium, or $r_{3,2}$. We can generalise this to $i-$number of grids as follows.

\begin{equation}
  r_{i+1,i} = \frac{S_{\text{grid},i+1}}{S_{\text{grid},i}},
  \label{eq:refinementRatio}
\end{equation}

\noindent where $S_{\text{grid},i}$ denotes the total number of cells in the $i^{\text{th}}$ grid. The GCI of $\clrms$ drops further to $1.63\%$ as the mesh is refined more with a refinement ratio of $1.304$.

The GCI for $\yrms$ also drops by one order of magnitude as can be seen by comparing $\text{GCI}_{3,2}$ with $\text{GCI}_{2,1}$. Again, this similar trend of improvement points to the causal relationship between lift and displacement of the cylinder. The GCI for $\fstr$, however, drops by approximately a factor of $6$ instead of one order of magnitude, unlike the GCIs of $\yrms$ and $\clrms$.

\begin{table}[width=0.9\linewidth,cols=4,pos=h]
  \caption{Summary of grid independency study.} \label{tab:gridIndependency}
\begin{tabular*}{\tblwidth}{@{} LLLL@{} }
\toprule
Parameter/ metric                                                       & $\clrms$       & $\yrms = \ystr/D$ & $\fstr = \fcyl / \fn$ \\
\midrule
$\fre$                                                                  & $0.262$        & $0.369$           & $0.969$               \\
$f_{1}$                                                                 & $0.2598$       & $0.3687$          & $0.9695$              \\
$f_{2}$                                                                 & $0.2430$       & $0.3588$          & $0.9740$              \\
$f_{3}$                                                                 & $0.0805$       & $0.2374$          & $1.0220$              \\
$\left | \epsilon_{2,1} \right |$                                       & $0.02$         & $0.01$            & $0.004$               \\
$\left | \epsilon_{2,1} \right |$                                       & $0.16$         & $0.12$            & $0.48$                \\
$R = \left | \epsilon_{2,1} \right | / \left | \epsilon_{2,1} \right |$ & $0.10$         & $0.08$            & $0.094$               \\
$\text{GCI}_{3,2}$                                                      & $30.92$        & $6.00$            & $0.64$                \\  
$\text{GCI}_{3,2}$                                                      & $1.63$         & $0.52$            & $0.10$                \\
\bottomrule
\end{tabular*}
\end{table}

We provide visual representations of the convergent $\clrms$, $\yrms$ and $\fstr$ series in Figs. \ref{fig:yrmsGCI}, \ref{fig:fstrGCI} and \ref{fig:clrmsGCI}. Note how the quantity of interest is very close to its Richardson extrapolation at the fine grid (grid 1) for all $\clrms$, $\yrms$ and $\fstr$. This implies that the fine grid already provides adequate spatial discretisation for the problem we are studying, and further refinements, while able to nudge our solutions even closer to the limit that is the Richardson extrapolation, may not be optimal in terms of usage of computational resources. Values of $\yrms$ and $\fstr$ at the fine grid already fall within experimental uncertainty as evidenced by our measurement in \S \ref{ssec:openFlowExp} and the work by \citet{Koide2013}. Hence, all succeeding numerical data are gathered from the fine grid.


\begin{figure}
  \centering
  \includegraphics[width=0.39\textwidth]{figs/figure7}
  \caption{The convergence diagram for $\yrms$. Figure \ref{fig:yrmsGCI}a shows how $\yrms$ converges close to the Richardson extrapolation value while Fig. \ref{fig:yrmsGCI}b shows how the error (difference between the value obtained from a particular mesh and the Richardson extrapolation) decreases with decreasing grid spacing.} \label{fig:yrmsGCI}
\end{figure}

\begin{figure}
  \centering
  \includegraphics[width=0.4\textwidth]{figs/figure8}
  \caption{The convergence diagram for $\fstr$. Figure \ref{fig:fstrGCI}a shows how $\fstr$ converges close to the Richardson extrapolation value while Fig. \ref{fig:fstrGCI}b shows how the error (difference between the value obtained from a particular mesh and the Richardson extrapolation) decreases with decreasing grid spacing.} \label{fig:fstrGCI}
\end{figure}

\begin{figure}
  \centering
  \includegraphics[width=0.43\textwidth]{figs/figure9}
  \caption{The convergence diagram for $\clrms$. Figure \ref{fig:clrmsGCI}a shows how $\clrms$ converges close to the Richardson extrapolation value while Fig. \ref{fig:clrmsGCI}b shows how the error (difference between the value obtained from a particular mesh and the Richardson extrapolation) decreases with decreasing grid spacing.} \label{fig:clrmsGCI}
\end{figure}

\section{Streamwise vortex-driven vibration regime}\label{sec:svivRegime}
The pure cruciform case, i.e. \SI{90}{\degree}, demonstrated a normalised \rms{} amplitude of cylinder displacement, $\yrms$ that starts quite expectedly with a low amplitude at reduced velocities $\uron$ and $\urtw$ , before reaching a value close to $\yrms = 0.1$ at $\ured = \urth$, as presented in Fig. \ref{fig:yStrRMSStreamwise}. Following the local maximum at $\ured = \urth$, $\yrms$ tapers off to less than $\yrms = 0.05$ between $\urfo \leq \yrms \leq \ursi$. This whole $\yrms$ trend of hitting a local maximum before tapering off bears a striking resemblance to the amplitude response of an isolated circular cylinder in KVIV at mass ratios of order $O(10^{1})$ \citep{Feng1963,Khalak1999}. This resemblance can be seen as an indication that the vibration of a pure cruciform between $\ured \leq \urse$ is driven primarily through the shedding cycle of Karman vortices.

\begin{figure}
  \centering
  \includegraphics[width=0.38\textwidth]{figs/yStrRMS1}
  \caption{Evolution of the normalised \rms{} amplitude of cylinder displacement $\yrms$, with respect to reduced velocity $\ured$, in the streamwise vortex-driven vibration regime.} \label{fig:fig:yStrRMSStreamwise}
\end{figure}

Then at $\ured = \urse$, $\yrms$ experiences a very weak increase followed by a sudden jump close to $0.4$ at $\ured = \urei$. This is followed by a slight decline at $\ured = \urni$ and return to the previous level of $\yrms$ at $\ured = \urte$. Past $\ured = \urte$, we observe that $\yrms$ maintains a linear trend in its variation with respect to $\ured$. As $\ured = \urei$ is well within the lower branch for a system in KVIV, it is quite unlikely for the vibration within $\urei \leq \ured \leq \urtt$ to be the governed by the shedding of Karman vortices, leading previous investigators to attribute the vibration to the periodic shedding of streamwise vortical structures dominating the spatial region close to the cruciform juncture \citep{Shirakashi1989,Hemsuwan2018b,Hemsuwan2018d}. Hence, we name this range of $\ured$ the streamwise vortex-induced vibration regime.

\begin{figure}
  \centering
  \includegraphics[width=0.4\textwidth]{figs/expCompareAmp}
  \caption{Comparison between the evolution of $\yrms$ with respect to $\ured$of a pure cruciform system from our numerical and experimental work. The filled square represents the numerical, while the filled circle represents the experimental results.}
  \label{fig:expCompareAmp}
\end{figure}

The experimental system consisting of the closed loop open flow channel and the pure cruciform oscillator rig in \S\ref{ssec:openFlowExp} is constructed not only for the purpose of validating the results of our pure cruciform numerical investigation, but also to corroborate in general, the sum total of our numerical setup. Admittedly, the best undertaking would be to perform equivalent experiment for each of the \SI{90}{\degree}, \SI{67.5}{\degree}, \SI{45}{\degree}, \SI{22.5}{\degree} and \SI{0}{\degree} configurations, but the scale of such an exercise and subsequent discussion of the results in our opinion, deserves its own treatment separate from the current study. The degree of agreement between the results of our numerical and experimental investigation of the pure cruciform establishes the validity of our numerical setup, which we assume to extend to the rest of the cruciforms. We think that this assumption is somewhat founded because all cruciforms are simulated under similar boundary conditions, mesh resolution and solver algorithm.

Our experiments collect time series data of cylinder displacement $y$, from which normalised \rms{} amplitude $\yrms$, and normalised cylinder vibration frequency $\fstr$, are computed. Figure \ref{fig:expCompareAmp} compares both our numerical and experimental results of $\yrms$. We observe that both results agree in terms of magnitude and trend of the amplitude response. However, the jump to SVIV occurs at a higher $\ured \approx 19$, translating to a delay of about 3 units of $\ured$. Our numerical and experimental results are also able to capture the slight dip in $\yrms$ following the jump to SVIV, but the occurrence in our experiment is also delayed by about 3 units of $\ured$. This delay can perhaps be attributed to the fact that the raw $y$ time series were measured in succession from the lowest attainable channel flow velocity \uth{} to its highest \uel{} within one experimental run. In contrast, our simulations always start with the cylinder at rest at its neutral position at $t_{0} = \SI{0}{\second}$, with the freestream exactly at set at the desired value $\uon, \utw, \dots, \utt$. Thus, the delays found in our experimental results may simply be the consequence of ``flow memory'', a concept whose analogy can be found in undergraduate experiments to determine the critical Reynolds number transitioning from laminar to turbulent flow in smooth circular pipes. The ``flow memory'' is of course the manifestation of flow inertia due to fluid viscosity, where the flow has a natural tendency to retain its previous state before being overpowered by the flow momentum. This results in the delay found at the jump to SVIV and the local $\yrms$ minimum after the jump.

\begin{figure}
  \centering
  \includegraphics[width=0.38\textwidth]{figs/yStrFreq5}
  \caption{Evolution of the normalised cylinder displacement frequency, $\fstr$, with respect to reduced velocity $\ured$, for the pure cruciform case.}
  \label{fig:yStrFreq5}
\end{figure}

We show the evolution of the normalised cylinder vibration frequency $\fstr$ with respect to $\ured$ in Fig. \ref{fig:yStrFreq5}. Inspecting Fig. \ref{fig:yStrFreq5}, we immediately notice two distinct evolutionary pattern for $\fstr$with a sharp boundary at $\ured = \ursi$. Between $\uron \leq \ured \leq \ursi$, the $\fstr$ trend follows closely the shedding frequency of Karman vortices from an isolated, fixed circular cylinder \citep{Blevins1990}. The Karman vortex shedding frequency is given as an empirical equation in Eq. \ref{eq:karmanSheddingFreq}.

\begin{equation}
  f_{v} = 0.198 \left( 1 - \frac{19.7}{\re} \right) DU
  \label{eq:karmanSheddingFreq}
\end{equation}

\noindent Here, $f_{v}$, $D$ and $U$ are the vortex shedding frequency, diameter of the isolated circular cylinder and $U$ the freestream velocity respectively. We can easily see how $f_{v}$ is a linear function of $U$, and this is what gives rise to the linear pattern of $\fstr$ within $\uron \leq \ured \leq \ursi$.

\begin{figure}
  \centering
  \begin{subfigure}[h]{0.38\textwidth}
    \includegraphics[width=\textwidth]{figs/clFreq5}
    \caption{Evolution of at $\alpha = \SI{90}{\degree}$.}
    \label{fig:clFreq5}
  \end{subfigure}

  \begin{subfigure}[h]{0.38\textwidth}
    \includegraphics[width=\textwidth]{figs/clFreq4}
    \caption{Amplitude response at $\alpha = \SI{67.5}{\degree}$.}
    \label{fig:clFreq4}
  \end{subfigure}
  \caption{Evolution of the normalised frequency of lift coefficient, $\fclstr$ with respect to reduced velocity $\ured$ in the streamwise vortex-driven regime.} \label{fig:yStrRMSStreamwise}
\end{figure}

\section{Transition to Karman vortex-driven vibration regime}\label{sec:suppRegime}
We then continue our exploration with \SI{45}{\degree}. And some more text.

\begin{figure}
  \centering
  \begin{subfigure}[h]{0.38\textwidth}
    \includegraphics[width=\textwidth]{figs/yStrRMS3}
    \caption{Tilt angle $\alpha = \SI{45}{\degree}$.}
    \label{fig:yStrRMS3}
  \end{subfigure}
  \caption{Evolution of the normalised RMS amplitude of cylinder displacement $\yrms$, with respect to reduced velocity $\ured$, in the region where Karman and streamwise vortex-driven vibrations are most suppressed.} \label{fig:yStrRMSSuppressed}
\end{figure}

\begin{figure}
  \centering
  \begin{subfigure}[h]{0.38\textwidth}
    \includegraphics[width=\textwidth]{figs/clRMS3}
    \caption{Evolution of at $\alpha = \SI{45}{\degree}$.}
    \label{fig:clRMS3}
  \end{subfigure}

  \caption{Evolution of the RMS amplitude of lift coefficient $\clrms$,  with respect to reduced velocity $\ured$, in the region where Karman and streamwise vortex-driven vibrations are most suppressed.} \label{fig:clRMSSuppressed}
\end{figure}

\begin{figure}
  \centering
  \begin{subfigure}[h]{0.38\textwidth}
    \includegraphics[width=\textwidth]{figs/clFreq3}
    \caption{Evolution of at $\alpha = \SI{45}{\degree}$.}
    \label{fig:clFreq3}
  \end{subfigure}

  \caption{Evolution of the normalised frequency of lift coefficient, $\fclstr$ with respect to reduced velocity $\ured$ in the region where Karman and streamwise vortex-induced vibrations are most suppressed.} \label{fig:clFreqSuppressed}
\end{figure}

\section{Karman vortex-driven vibration regime}\label{sec:enhKarmanRegime}

\begin{figure}
  \centering
  \begin{subfigure}[h]{0.35\textwidth}
    \includegraphics[width=\textwidth]{figs/yStrFreq2}
    \caption{Tilt angle $\alpha = \SI{22.5}{\degree}$.}
    \label{fig:yStrFreq2}
  \end{subfigure}

  \begin{subfigure}[h]{0.35\textwidth}
    \includegraphics[width=\textwidth]{figs/yStrFreq1}
    \caption{Tilt angle $\alpha = \SI{0}{\degree}$.}
    \label{fig:yStrFreq1}
  \end{subfigure}
  \caption{Evolution of the normalised cylinder displacement frequency, $\fstr$, with respect to reduced velocity $\ured$, in the Karman vortex-driven vibration regime.} \label{fig:yStrFreqKarman}
\end{figure}

\begin{figure}
  \centering
  \begin{subfigure}[h]{0.38\textwidth}
    \includegraphics[width=\textwidth]{figs/clFreq2}
    \caption{Evolution of at $\alpha = \SI{45}{\degree}$.}
    \label{fig:clFreq2}
  \end{subfigure}

  \begin{subfigure}[h]{0.38\textwidth}
    \includegraphics[width=\textwidth]{figs/clFreq1}
    \caption{Evolution of at $\alpha = \SI{45}{\degree}$.}
    \label{fig:clFreq1}
  \end{subfigure}

  \caption{Evolution of the normalised frequency of the lift coefficient,  with respect to reduced velocity $\ured$ in the Karman vortex-driven regime.} \label{fig:clFreqKarman}
\end{figure}

\section{Power characteristic in $\alpha$ -- $\ured$ parameter space}\label{sec:powerCharacteristic}

\begin{figure}
  \centering
  \includegraphics[width=0.38\textwidth]{figs/yRMSContour}
  \caption{Isocontours describing the map of the normalised RMS amplitude of cylinder displacment, $\yrms$ in the cruciform angle - reduced velocity ($\alpha$--$\ured$) parameter space.}
  \label{fig:yRMSContour}
\end{figure}

\begin{figure}
  \centering
  \includegraphics[width=0.38\textwidth]{figs/mechanicalPowerContours}
  \caption{Isocontours describing the map of the estimated mechanical power in the cruciform angle - reduced velocity ($\alpha$--$\ured$) parameter space.}
  \label{fig:mechanicalPowerContour}
\end{figure}

\begin{figure}
  \centering
  \includegraphics[width=0.39\textwidth]{figs/powerEfficiencyContours}
  \caption{Isocontours describing the map of the estimated mechanical power in the cruciform angle - reduced velocity ($\alpha$--$\ured$) parameter space.}
  \label{fig:powerEfficiencyContour}
\end{figure}

\section{Conclusions} \label{sec:conclusions}
In this study, we numerically investigated the temporal evolution of the lift coefficient and cylinder displacement signals of an elastically supported cruciform system in the range $1.1 \times 10^{3} < \re < 14.6 \times 10^{3}$, or $\uron < \ured < \urtt$. Our circular cylinder diameter is \SI{10}{\milli\metre} and the natural frequency of the system is \SI{4.4}{\hertz}. Validation of key numerical results was made experimentally in a custom-built open flow channel, using a cruciform system whose parameters were tuned as close as possible to the quantities used in the numerical study. Decomposing the lift coefficient signal in the SVIV regime ($\urse \leq \ured \leq \urtt$) using EEMD allows us to see that the complexity of the lift coefficient signal as being caused by the superpositioning of two dominant components of lift. One due to the shedding of Karman and the other due to the shedding of streamwise vortices. The former has a frequency close to the vortex shedding frequency of Karman vortex from a smooth, isolated circular cylinder, while the latter has a mean frequency close to $\fn$. Application of the Hilbert-Huang transform on the dominant component of cylinder displacement -- and the component of lift most correlated to it -- allows for the computation of the instantaneous phase lag between lift and cylinder displacement. The time-averaged phase lag revealed five ``branches'' of vibration, among which is the initial branch of SVIV at $\ured = \urei$, which has never been identified before in the literature. We also computed the instantaneous frequency of the lift coefficient, thus revealing the loss of periodicity and self-similarity in the lift coefficient signal as the system enters the SVIV regime. Estimation of power from our results show that the \rms{} mechanical and fluid power computed from our experimental and numerical work agree to varying degrees depending on $\ured$ with data from similar studies in the literature. Finally, we estimated that the \rms{} fluid power can potentially be increased close to a factor of 2 within $\urei \leq \ured \leq \urte$ and close to a factor of 3 when $\urel \leq \ured \leq \urtt$. We base this estimation on the premise of redirecting the contribution to the \rms{} amplitude of total lift from Karman vortex shedding, towards the streamwise component of lift alone.

%\appendix
%\section{Appendix}
%Appendix sections are coded under \verb+\appendix+.
%
%\verb+\printcredits+ command is used after appendix sections to list 
%author credit taxonomy contribution roles tagged using \verb+\credit+ 
%in frontmatter.
%

\printcredits

%% Loading bibliography style file
%\bibliographystyle{model1-num-names}
\bibliographystyle{cas-model2-names}

% Loading bibliography database
\bibliography{references}


%\vskip3pt

%\bio{}
%Author biography without author photo.
%Author biography. Author biography. Author biography.
%Author biography. Author biography. Author biography.
%Author biography. Author biography. Author biography.
%Author biography. Author biography. Author biography.
%Author biography. Author biography. Author biography.
%Author biography. Author biography. Author biography.
%Author biography. Author biography. Author biography.
%Author biography. Author biography. Author biography.
%Author biography. Author biography. Author biography.
%\endbio
%
%\bio{figs/pic1}
%Author biography with author photo.
%Author biography. Author biography. Author biography.
%Author biography. Author biography. Author biography.
%Author biography. Author biography. Author biography.
%Author biography. Author biography. Author biography.
%Author biography. Author biography. Author biography.
%Author biography. Author biography. Author biography.
%Author biography. Author biography. Author biography.
%Author biography. Author biography. Author biography.
%Author biography. Author biography. Author biography.
%\endbio
%
%\bio{figs/pic1}
%Author biography with author photo.
%Author biography. Author biography. Author biography.
%Author biography. Author biography. Author biography.
%Author biography. Author biography. Author biography.
%Author biography. Author biography. Author biography.
%\endbio

\end{document}
